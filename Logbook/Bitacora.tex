\documentclass[12pt,a4paper]{report}
\usepackage[latin1]{inputenc}
\usepackage[spanish,es-tabla]{babel}
\usepackage{graphicx}
\usepackage[left=3cm,right=3cm,top=2.5cm,bottom=2.5cm]{geometry}
\usepackage{lastpage}
\usepackage{fancyhdr}
\pagestyle{fancy}
\fancyhead[R]{\textbf{\thepage/\pageref{LastPage}}}
\renewcommand{\headrulewidth}{0pt}

\begin{document}
\begin{titlepage}
\begin{center}
\vspace*{1.5cm}
\textbf{Escuela de Ingenier�a en Electr�nica}\\[0.8cm]
\textbf{Laboratorio de Dise�o de Sistemas Digitales}\\[1cm]
\textbf{Bit�cora}\\[2cm]
\textbf{Proyecto:}\\[0.4cm]
Control y programaci�n RTC con Nexys3 \\[1.7cm]
\textbf{Profesor:}\\[0.4cm]
Alfonso Chac�n Rodr�guez \\[1.7cm]
\textbf{Estudiantes:}\\[0.4cm]
Keylor Mena Venegas \\[0.8cm]
Luis Leon Vega \\[0.8cm]
Luis Merayo Gatica \\[1.7cm]
\textbf{Periodo}\\[0.8cm]
II Semestre, 2016\\
\end{center}
\end{titlepage}

\begin{flushright}
\begin{large}
\textbf{Fecha: 4 Mayo 2016}\\
\end{large}
\end{flushright}

\section*{\textit{Descripci�n del problema}}

Lorem ipsum dolor sit amet, consectetur adipiscing elit. Ut semper imperdiet magna id sollicitudin. Curabitur sollicitudin posuere diam at lobortis. Nam a elit ut neque suscipit consectetur vel ac diam. Etiam purus tortor, consectetur vel lacus in, luctus condimentum libero. Cras efficitur luctus augue quis lobortis. Aliquam erat volutpat. Donec lobortis velit sit amet ante rutrum, sed vestibulum sem ultricies. Nullam accumsan accumsan massa, non vehicula urna auctor et. \\[2ex]

Maecenas accumsan vestibulum bibendum. Phasellus a tortor magna. Phasellus vulputate dignissim blandit. Curabitur non nisi velit. Vivamus accumsan quam eget dui imperdiet, et finibus nisi placerat. Nulla laoreet rhoncus pellentesque. Duis lacinia eget nisl vel fringilla. \\[2ex]

\section*{\textit{Introducci�n al proyecto}}

Lorem ipsum dolor sit amet, consectetur adipiscing elit. Ut semper imperdiet magna id sollicitudin. Curabitur sollicitudin posuere diam at lobortis. Nam a elit ut neque suscipit consectetur vel ac diam. Etiam purus tortor, consectetur vel lacus in, luctus condimentum libero. Cras efficitur luctus augue quis lobortis. Aliquam erat volutpat. Donec lobortis velit sit amet ante rutrum, sed vestibulum sem ultricies. Nullam accumsan accumsan massa, non vehicula urna auctor et.\\[2ex]

Maecenas accumsan vestibulum bibendum. Phasellus a tortor magna. Phasellus vulputate dignissim blandit. Curabitur non nisi velit. Vivamus accumsan quam eget dui imperdiet, et finibus nisi placerat. Nulla laoreet rhoncus pellentesque. Duis lacinia eget nisl vel fringilla. \\[2ex]

\section*{\textit{Objetivo General}}
Maecenas accumsan vestibulum bibendum. Phasellus a tortor magna. Phasellus vulputate dignissim blandit. Curabitur non nisi velit. \\[2ex]

\section*{\textit{Objetivos Espec�ficos}}
\begin{itemize}
\item Maecenas accumsan vestibulum bibendum. Phasellus a tortor magna
\item Maecenas accumsan vestibulum bibendum. Phasellus a tortor magna
\end{itemize}

\newpage

% Comienzo de la bitacora
\section*{\textit{Control de eventos}}

% Nueva entrada
\begin{flushright}
	\begin{large}
		\textbf{Fecha: 30 de Agosto}\\[5ex]
	\end{large}
\end{flushright}

\noindent \textbf{Integrantes:} Luis Leon, Luis Merayo, Keylor Mena \\[1ex]
\textbf{Hora:} 13:00 - 15:00 \\[1ex]
\textbf{Actividad:} \\[2ex]

El profesor explic� el instructivo y expres� las necesidades que deben ser cubiertas durante todo el proyecto. Qued� claro en que se debe desarrollar unidades de control para el proyecto (que se pueden hacer con FSM o M�quinas de Estados Finita) y desarrollar un controlador para obtener datos del RTC y para programarlo. \\

t El reto de este �ltimo surge, principalmente, de comunicar el circuito al RTC mediante un bus multiplexado de direcciones y datos (A/D Bus). Se debe crear un control que permita escribir y leer de forma controlada y, con ello, que se pueda seleccionar el hardware para cada caso. \\

\indent En este d�a, se hizo una reuni�n grupal para decidir las tareas de cada miembro, lo cual estableci�: 

\begin{itemize}
	\item Luis Leon: Control general del circuito - Controlador VGA
	\item Luis Merayo: Control del RTC
	\item Keylor Mena: Encargado de testbench.
\end{itemize}

No se estableci� ninguna aproximaci�n al dise�o del proyecto para dar tiempo de que cada miembro pueda estudiar el instructivo y el RTC con calma y de forma individual. Asimismo, se qued� que el d�a de ma�ana, esta tarea ser� efectuada al igual que la distribuci�n de todas las tareas que puede implicar el proyecto. \\

Sin embargo, se ha decido usar el control VGA que se ha dise�ado en el proyecto anterior para agilizar la tarea del nuevo dise�o de proyecto, asumiendo de que se conocen todas las partes de este controlador. Por otro lado, para efectos de la comprensi�n del nuevo proyecto, se asisti� a la tutor�a el d�a anterior, brindando la facilidad de que uno de los miembros conozca el sistema de antemano. Para ello ser� recomendado ver el enlace de inter�s al final de esta entrada. \\

\noindent \textbf{Enlaces de inter�s:}
\begin{itemize}
	
	\item Documento de la tutor�a: http://bit.ly/2bEI3RZ
	\item Datasheet del RTC: http://bit.ly/2cpNGJa
\end{itemize}


\end{document}